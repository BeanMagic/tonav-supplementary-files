\chapter*{Introduction}
\addcontentsline{toc}{chapter}{Introduction}

1 to 3 pages

describe problem, structure at the end


\section{Cíl práce}

Na konci by měl být program, který je schopný dělat lokalizaci pomocí IMU a videa z kamery s kompenzací rolling shutter. Navíc by SW měl být schopen kompenzovat naakumulovanou chybu tím, že si bude tvořit interní mapu a v ní pomocí loop closure opravovat chyby. 

Další požadavky:
\begin{itemize}
\item Bude to balíček do ROSu (i když vývoj bude pravděpodobně probíhat jinak)
\item Otevřená licence
\item Musíme být schopni snadno ověřit výsledky (nějak snadno porovnatelné s ground truth)
\item Celá lokalizace by měla být energeticky nenáročná
\item Kalibrace by neměla vyžadovat žádné speciální znalosti (například časy uzávěrky apod.)
\end{itemize}

\section{Co očekávám, že bude moje přidaná hodnota}

Největší "akademická přidaná hodnota" by mělo být přidání globální mapy a loop closure k metodě založené na IMU. To pokud ví, nikdo nemá.

Další velká přidaná hodnota bude v otevřenosti samotného řešení. Opět, asi v současné době nejsou naimplementované žádné open-source řešení, která by fungovala takto sofistikovaně i s kamerou atp.

Integrace různých senzorů, stačí popsat jak se to udělá na nějaké hrubší úrovni.